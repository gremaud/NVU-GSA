% % % % % % % % % % % % % % % % % % % % % % % 
%
% Once your paper is accepted for publication, 
% PLEASE REMOVE ALL TRACKED CHANGES in this file 
% and leave only the final text of your manuscript. 
% PLOS recommends the use of latexdiff to track changes during review, as this will help to maintain a clean tex file.
% Visit https://www.ctan.org/pkg/latexdiff?lang=en for info or contact us at latex@plos.org.
%
%
% There are no restrictions on package use within the LaTeX files except that 
% no packages listed in the template may be deleted.
%
% % % % % % % % % % % % % % % % % % % % % % %
%
% Tables should be cell-based and may not contain:
% - spacing/line breaks within cells to alter layout or alignment
% - do not nest tabular environments (no tabular environments within tabular environments)
% - no graphics or colored text (cell background color/shading OK)
% See http://journals.plos.org/plosone/s/tables for table guidelines.
%
% For tables that exceed the width of the text column, use the adjustwidth environment as illustrated in the example table in text below.
%
% % % % % % % % % % % % % % % % % % % % % % % %
%
% -- EQUATIONS, MATH SYMBOLS, SUBSCRIPTS, AND SUPERSCRIPTS
%
% IMPORTANT
% Below are a few tips to help format your equations and other special characters according to our specifications. For more tips to help reduce the possibility of formatting errors during conversion, please see our LaTeX guidelines at http://journals.plos.org/plosone/s/latex
%
% For inline equations, please be sure to include all portions of an equation in the math environment.  For example, x$^2$ is incorrect; this should be formatted as $x^2$ (or $\mathrm{x}^2$ if the romanized font is desired).
%
% Do not include text that is not math in the math environment. For example, CO2 should be written as CO\textsubscript{2} instead of CO$_2$.
%
% Please add line breaks to long display equations when possible in order to fit size of the column. 
%
% For inline equations, please do not include punctuation (commas, etc) within the math environment unless this is part of the equation.
%
% When adding superscript or subscripts outside of brackets/braces, please group using {}.  For example, change "[U(D,E,\gamma)]^2" to "{[U(D,E,\gamma)]}^2". 
%
% Do not use \cal for caligraphic font.  Instead, use \mathcal{}
%
% % % % % % % % % % % % % % % % % % % % % % % % 
%
% Please contact latex@plos.org with any questions.
%
% % % % % % % % % % % % % % % % % % % % % % % %

\documentclass[10pt,letterpaper]{article}
\usepackage[top=0.85in,left=2.75in,footskip=0.75in]{geometry}
\usepackage{amsmath,amssymb}
\usepackage{changepage}
\usepackage[utf8x]{inputenc}
\usepackage{textcomp,marvosym}
\usepackage{cite}
\usepackage{nameref,hyperref}
\usepackage[right]{lineno}
\usepackage{microtype}
\DisableLigatures[f]{encoding = *, family = * }

% color can be used to apply background shading to table cells only
\usepackage[table]{xcolor}
\usepackage{array}

% create "+" rule type for thick vertical lines
\newcolumntype{+}{!{\vrule width 2pt}}

% create \thickcline for thick horizontal lines of variable length
\newlength\savedwidth
\newcommand\thickcline[1]{%
	\noalign{\global\savedwidth\arrayrulewidth\global\arrayrulewidth 2pt}%
	\cline{#1}%
	\noalign{\vskip\arrayrulewidth}%
	\noalign{\global\arrayrulewidth\savedwidth}%
}

% \thickhline command for thick horizontal lines that span the table
\newcommand\thickhline{\noalign{\global\savedwidth\arrayrulewidth\global\arrayrulewidth 2pt}%
	\hline
	\noalign{\global\arrayrulewidth\savedwidth}}


% Remove comment for double spacing
%\usepackage{setspace} 
%\doublespacing

% Text layout
\raggedright
\setlength{\parindent}{0.5cm}
\textwidth 5.25in 
\textheight 8.75in

% Bold the 'Figure #' in the caption and separate it from the title/caption with a period
% Captions will be left justified
\usepackage[aboveskip=1pt,labelfont=bf,labelsep=period,justification=raggedright,singlelinecheck=off]{caption}
\renewcommand{\figurename}{Fig}

\bibliographystyle{plos2015}

% Remove brackets from numbering in List of References
\makeatletter
\renewcommand{\@biblabel}[1]{\quad#1.}
\makeatother

% Header and Footer with logo
\usepackage{lastpage,fancyhdr,graphicx}
\usepackage{epstopdf}
%\pagestyle{myheadings}
\pagestyle{fancy}
\fancyhf{}
%\setlength{\headheight}{27.023pt}
%\lhead{\includegraphics[width=2.0in]{PLOS-submission.eps}}
\rfoot{\thepage/\pageref{LastPage}}
\renewcommand{\headrulewidth}{0pt}
\renewcommand{\footrule}{\hrule height 2pt \vspace{2mm}}
\fancyheadoffset[L]{2.25in}
\fancyfootoffset[L]{2.25in}
\lfoot{\today}

%% Include all macros below

\newcommand{\lorem}{{\bf LOREM}}
\newcommand{\ipsum}{{\bf IPSUM}}

%% END MACROS SECTION


\begin{document}
	\vspace*{0.2in}
	\begin{flushleft}
		{\Large
			\textbf\newline{Title title title title title title title title \textinterrobang 250 characters max} 
		}
		\newline
		% Insert author names, affiliations and corresponding author email (do not include titles, positions, or degrees).
		\\
		R.M.P. Morillo \textsuperscript{1*},
		Tim David \textsuperscript{2},
		Pierre A. Greamaud \textsuperscript{1}
		\\
		\bigskip
		\textbf{1} Department of Mathematics, North Carolina State University, Raleigh, North Carolina, United States of America
		\\
		\textbf{2} Department of Mechanical Engineering, University of Canterbury, New Zealand
		\\
		\bigskip
		
		% Insert additional author notes using the symbols described below. Insert symbol callouts after author names as necessary.
		% 
		% Remove or comment out the author notes below if they aren't used.
		%
		% Primary Equal Contribution Note
		%\Yinyang These authors contributed equally to this work.
		
		% Additional Equal Contribution Note
		% Also use this double-dagger symbol for special authorship notes, such as senior authorship.
		%\ddag These authors also contributed equally to this work.
		
		% Current address notes
		%\textcurrency Current Address: Dept/Program/Center, Institution Name, City, State, Country % change symbol to "\textcurrency a" if more than one current address note
		% \textcurrency b Insert second current address 
		% \textcurrency c Insert third current address
		
		% Deceased author note
		%\dag Deceased
		
		% Group/Consortium Author Note
		%\textpilcrow Membership list can be found in the Acknowledgments section.
		
		% Use the asterisk to denote corresponding authorship and provide email address in note below.
		* rmmorill@ncsu.edu
		
	\end{flushleft}
	
	\section*{\textinterrobang Notes that came in the template}
	Please do not include colors or graphics in the text.
	
	The manuscript LaTeX source should be contained within a single file (do not use \verb*|\input|, \verb*|\externaldocument|, or similar commands).
	
	DO NOT INCLUDE GRAPHICS IN YOUR MANUSCRIPT (more details commented out)
	% - Figures should be uploaded separately from your manuscript file. 
	% - Figures generated using LaTeX should be extracted and removed from the PDF before submission. 
	% - Figures containing multiple panels/subfigures must be combined into one image file before submission.
	% For figure citations, please use "Fig" instead of "Figure".
	% See http://journals.plos.org/plosone/s/figures for PLOS figure guidelines.
	
	Please use "sentence case" for title and headings (capitalize only the first word in a title (or heading), the first word in a subtitle (or subheading), and any proper nouns).
	
	Use "Eq" instead of "Equation" for equation citations.
	
	For figure citations, please use "Fig" instead of "Figure".
	
	Place figure captions after the first paragraph in which they are cited. (example figure commented out)
	%\begin{figure}[!h]
	%	\caption{{\bf Bold the figure title.}
	%		Figure caption text here, please use this space for the figure panel descriptions instead of using subfigure commands. A: Lorem ipsum dolor sit amet. B: Consectetur adipiscing elit.}
	%	\label{fig1}
	%\end{figure}
	
	Place tables after the first paragraph in which they are cited. (example table commented out)
	%\begin{table}[!ht]
	%	\begin{adjustwidth}{-2.25in}{0in} % Comment out/remove adjustwidth environment if table fits in text column.
	%		\centering
	%		\caption{
	%			{\bf Table caption Nulla mi mi, venenatis sed ipsum varius, volutpat euismod diam.}}
	%		\begin{tabular}{|l+l|l|l|l|l|l|l|}
	%			\hline
	%			\multicolumn{4}{|l|}{\bf Heading1} & \multicolumn{4}{|l|}{\bf Heading2}\\ \thickhline
	%			$cell1 row1$ & cell2 row 1 & cell3 row 1 & cell4 row 1 & cell5 row 1 & cell6 row 1 & cell7 row 1 & cell8 row 1\\ \hline
	%			$cell1 row2$ & cell2 row 2 & cell3 row 2 & cell4 row 2 & cell5 row 2 & cell6 row 2 & cell7 row 2 & cell8 row 2\\ \hline
	%			$cell1 row3$ & cell2 row 3 & cell3 row 3 & cell4 row 3 & cell5 row 3 & cell6 row 3 & cell7 row 3 & cell8 row 3\\ \hline
	%		\end{tabular}
	%		\begin{flushleft} Table notes Phasellus venenatis, tortor nec vestibulum mattis, massa tortor interdum felis, nec pellentesque metus tortor nec nisl. Ut ornare mauris tellus, vel dapibus arcu suscipit sed.
	%		\end{flushleft}
	%		\label{table1}
	%	\end{adjustwidth}
	%\end{table}
	
	PLOS does not support heading levels beyond the 3rd (no 4th level headings).
	%\subsection*{\lorem\ and \ipsum\ nunc blandit a tortor}
	%\subsubsection*{3rd level heading} 
	
	\section*{Abstract}
	\textinterrobang max 300 words\\
	\textinterrobang no citations, minimal abbreviations if any\\
	
	The abstract. Summary of what we did, what we got, etc.
	
	
	% Please keep the Author Summary between 150 and 200 words
	% Use first person. PLOS ONE authors please skip this step. 
	% Author Summary not valid for PLOS ONE submissions.   
	%\section*{Author summary}

	
	\linenumbers
	
	
	\section*{Introduction}
	
	Very similar to the previous introduction: talk about the error propagation V model discrepancy curve,  (briefly) about what our model is, why we want to ``improve'' the model, and define notation for a generic model.
	
	Possibly also introduce screening, surrogate modeling, and sobol indices here too? So we have this section have all the math for a generic model and the methods section is about combining them and using them on a specific model?
	
	\textinterrobang Majority of citations here 
	
	\textinterrobang Note any relevant controversies or disagreements in the field?
	
	Conclude with a brief statement of the overall aim of the work and a comment about whether that aim was achieved
	
	\section*{Materials and methods}
	
	Introduce specific information about the model
	
	Introduce the specific data we have
	
	Define exactly what our QoI is, what we are trying to minimize, etc. 
	
	Discuss the added ``0th" step of building parameter distributions due to our specific model's behavior
	
	Discuss our specific choice of surrogate
	
	Discuss optimization specifics 
	
	
	
	
	% Results and Discussion can be combined.
	\section*{Results}
	
	
	Very similar to previous paper - show overall improvement and talk about how few parameters were moved and how little they were moved.
	
	Possibly talk about convergence of parameter distributions? Possibly talk about intermediate optimization results?
	
	
	\section*{Discussion}
	
	Same as before - talk about what this helps us know about the model and how doing OAT optimization would have moved some parameters in the wrong way
	
	
	\section*{Conclusion}
	
	Its a conclusion. Wrap everything up, re state key things obtained
	
	%For more information, see \nameref{S1_Appendix}.
	
	\section*{Supporting information}
	
	% Include only the SI item label in the paragraph heading. Use the \nameref{label} command to cite SI items in the text.
	\paragraph*{S1 Fig.}
	\label{S1_Fig}
	{\bf Bold the title sentence.} Add descriptive text after the title of the item (optional).
	
	\paragraph*{S2 Fig.}
	\label{S2_Fig}
	{\bf Lorem ipsum.} Maecenas convallis mauris sit amet sem ultrices gravida. Etiam eget sapien nibh. Sed ac ipsum eget enim egestas ullamcorper nec euismod ligula. Curabitur fringilla pulvinar lectus consectetur pellentesque.
	
	\paragraph*{S1 File.}
	\label{S1_File}
	{\bf Lorem ipsum.}  Maecenas convallis mauris sit amet sem ultrices gravida. Etiam eget sapien nibh. Sed ac ipsum eget enim egestas ullamcorper nec euismod ligula. Curabitur fringilla pulvinar lectus consectetur pellentesque.
	
	\paragraph*{S1 Video.}
	\label{S1_Video}
	{\bf Lorem ipsum.}  Maecenas convallis mauris sit amet sem ultrices gravida. Etiam eget sapien nibh. Sed ac ipsum eget enim egestas ullamcorper nec euismod ligula. Curabitur fringilla pulvinar lectus consectetur pellentesque.
	
	\paragraph*{S1 Appendix.}
	\label{S1_Appendix}
	{\bf Lorem ipsum.} Maecenas convallis mauris sit amet sem ultrices gravida. Etiam eget sapien nibh. Sed ac ipsum eget enim egestas ullamcorper nec euismod ligula. Curabitur fringilla pulvinar lectus consectetur pellentesque.
	
	\paragraph*{S1 Table.}
	\label{S1_Table}
	{\bf Lorem ipsum.} Maecenas convallis mauris sit amet sem ultrices gravida. Etiam eget sapien nibh. Sed ac ipsum eget enim egestas ullamcorper nec euismod ligula. Curabitur fringilla pulvinar lectus consectetur pellentesque.
	
	\section*{Acknowledgments}
	
	
	\nolinenumbers
	
	\section*{Author Contributions}
	\textinterrobang Not all needed, these are all the categories listed on the website \\\noindent
	\textbf{Conceptualization} \\\noindent
	\textbf{Data Curation} \\\noindent
	\textbf{Formal Analysis} \\\noindent
	\textbf{Funding Acquisition} \\\noindent
	\textbf{Investigation} \\\noindent
	\textbf{Methodology} \\\noindent
	\textbf{Project Administration} \\\noindent
	\textbf{Resources} \\\noindent
	\textbf{Software} \\\noindent
	\textbf{Supervision} \\\noindent
	\textbf{Validation} \\\noindent
	\textbf{Visualization} \\\noindent
	\textbf{Writing – Original Draft Preparation} \\\noindent
	\textbf{Writing – Review \& Editing} \\\noindent
	
	
	% Either type in your references using
	% \begin{thebibliography}{}
	% \bibitem{}
	% Text
	% \end{thebibliography}
	%
	% or
	%
	% Compile your BiBTeX database using our plos2015.bst
	% style file and paste the contents of your .bbl file
	% here. See http://journals.plos.org/plosone/s/latex for 
	% step-by-step instructions.
	% 
	\begin{thebibliography}{10}
		
		\bibitem{bib1}
		Conant GC, Wolfe KH.
		\newblock {{T}urning a hobby into a job: how duplicated genes find new
			functions}.
		\newblock Nat Rev Genet. 2008 Dec;9(12):938--950.
		
		\bibitem{bib2}
		Ohno S.
		\newblock Evolution by gene duplication.
		\newblock London: George Alien \& Unwin Ltd. Berlin, Heidelberg and New York:
		Springer-Verlag.; 1970.
		
		\bibitem{bib3}
		Magwire MM, Bayer F, Webster CL, Cao C, Jiggins FM.
		\newblock {{S}uccessive increases in the resistance of {D}rosophila to viral
			infection through a transposon insertion followed by a {D}uplication}.
		\newblock PLoS Genet. 2011 Oct;7(10):e1002337.
		
	\end{thebibliography}
	
	
	
\end{document}

